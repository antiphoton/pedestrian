\documentclass{article}
\usepackage[numbers]{natbib}
\usepackage{graphicx}
\bibliographystyle{apsrev}
\begin{document}
	\title{Pedestrain Modeling and Materials Science}
	\author{Boxiao Cao}
	\maketitle
	\section{Introduction}
		The social force model of pedestrain has been investigated \cite{helbing1995social,helbing1998generalized},
		and some self-organized pedestrain traffic has been simulated \cite{helbing2005self}.

		However, in some cases, this model still needs to be revised. In this work, some new potentials are introduced, calibrated and validated.

	\section{Problem Statement}

		\begin{tabular}{cc}
			\includegraphics[width=5cm]{1.ps} & \includegraphics[width=5cm]{2.ps} \\
			Simulation & Reality \\
		\end{tabular}

		Pedestrian would be self organized to lines rather than crowding like a semicircle.
		Is it possible to simulate this human-like behavior? What kind of potentials should be used?
		When would pedestrain no longer form lines?

	\section{Method}
		The programs are writen in C++.
	\renewcommand\refname{Reference}
	\bibliography{main}
\end{document}
